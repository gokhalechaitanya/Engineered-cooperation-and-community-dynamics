While this simple model is able to capture the essential dynamics it can be further reduced to reflect the minimal required parameters to reproduce the interactions.

For a general hypercycle we can have the intrinsic growth rates of $s_i$ for each of the individual species $i$ and $k_i$ as the feedback from the previous species in the cycle $x_{i-1}$.
We can model this dynamic as a version of the replicator mutator equation, which is a limiting case of a number of different ecological models e.g. with a normalised carrying capacity.

We modeled the population dynamics of each community as a replicator-mutator system.
The frequencies of the different strains are given by $x_i$ and their Malthusian growth rates are $f_i$.
The standard replicator-mutator equation is then denoted by \citep{nowak:TREE:1992,nowak:book:2006}:

\begin{equation}
	\dot{x}_i = \sum_{j=0}^{n} r_{j \rightarrow i} x_j f_j - x_i \phi.
\end{equation}
In this setup $\phi$ is defined so that the total population size of the two reactants remains constant.
The fitness of type $i$ in this case is assumed to be a frequency dependent quantity denoted by $f_i = 1$. Collating all these growth rates in a matrix $\mathbf{Q}_{ij}$

\begin{ali}
	\mathbf{Q}_{i,j} =
 \begin{pmatrix}
  r_{1 \rightarrow 1} & r_{2 \rightarrow 1} & \cdots & r_{n \rightarrow 1} \\
  r_{1 \rightarrow 2} & r_{2 \rightarrow 2} & \cdots & r_{n \rightarrow 2} \\
  \vdots  & \vdots  & \ddots & \vdots  \\
  r_{1 \rightarrow n} & r_{2 \rightarrow n} & \cdots & r_{n \rightarrow n} 
 \end{pmatrix}
\end{equation}
The average fitness of the population is given by $\phi = \sum_{j=0}^{n} x_j f_j$

For a single inoculum of the strains, the equations reduce to

\begin{align}
	\dot{x}_i &=  s_i x_i - x_i
\end{align}

\subsection{Two reactant system}

For a simple system with just two types we can define the dynamics by,

\begin{align}
	\dot{x}_1 &=  x_1 (s_1 x_1 + k_1 x_2 - \phi) \nonumber \\
	\dot{x}_2 &=  x_2 (k_2 x_1 + s_2 x_2 - \phi).
\end{align}
We denote $s_i$ as the self-replication rate of type $i$ and $k_i$ as the rate at which the reactant is generated by the reactant $i-1$.

\begin{itemize}
	\item Get basic growth rates of individual strains in complete media
	\item co-cultures in complete media
	\item co-cultures in reduced media
	\item competition experiment between communities (compete for space/other resources)
	\item modulate environment to see where they break down
	\item big exp
\end{itemize}