\section{Results and Discussion}

\begin{itemize}
	\item Evolving mutualism for industry - microbes are often found in a consortia rather than individual species therefore also for the processes such as watewater treatment, biofuel generation, oil spill cleanups etc., it might be possible to use symbiotic communities of bacteria rather than individual species \citep{zuroff:AppMB:2012}.
	\item Hypercycles were first describe as a model of pre-biotic evolution. Cooperation on the other hand can be assumed to be instrumental also from pre-biotic to contemporary timescales. When life colonises novel environments, it is hard to find it in isolation. Communities of bacteria are usually the pioneers in harsh environments from hydrothermal vents, lava flows and deep buried lava tubes. A few members of the consortia survive feeding of the inorganic matter but most of the community survives by a cross-feeding network.
\end{itemize}

\textbf{Acknowledgements} . All authors acknowledge the generous funding from the Max Planck Society.

\bibliographystyle{plainnat}