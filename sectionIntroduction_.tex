The major evolutionary transitions were not just the culmination of the Darwinian forces of mutation and selection. 
Cooperation plays a major role in binding together a level of selection so that it is possible for the ensemble of entities at one of the levels to be the target of selection on the higher level %\citep{maynard-smith:book:1995a}.
Genes to cells, cells to organisms and organisms to communities are some of the levels of selection where such a dynamic is observed.
It appears that when some form of conflict mediation is present resolving the issue of what is good for the group as opposed to that of the individual, selection can focus on the next level of organisation.
While numerous systems from RNA networks to human societies show cooperative behaviour, the trait needs explanation.
On first glance it seems to violate the survival of the fittest argument.
The conundrum can be explained via a number of compensatory mechanisms, ranging from the conceptual %\citep{nowak:Science:2006} 
to mechanistic arguments in mutualisms %\citep{akcay:bookchapter:2015}.
In societies, cooperation is often enforced via an external entity that controls the levels of punishment or reward. 
Such delegation of management to a third party can emerge through rudimentary rules on behaviour %\citep{sigmund:Nature:2010}.
In bacterial communities though, synthetic cooperation can be induced via genetic manipulation %\citep{shou:elife:2015,campbell:elife:2015}.
In nutritionally depleted environments such as recently exposed volcanic material, communities of microbes are the pioneers facilitating the further growth of life %\citep{kelly:MicEco:2014,fujimura:SciRep:2016}.
The fact that such pioneers are often \textit{communities} rather than \textit{individual} species,  critical in mediating biogeochemical cycles, bring forth the importance of cooperation right at the beginning of how life spread.

Why we need a simplified system

Ecological networks are complex.
Disentangling and understanding this complexity can be done at an abstract level employing information theory or network analysis.
While this provides us with general properties of the interactions it is often impossible to pin down the principles which form the building blocks.
One approach is to start with a simpler system with defined nature and number of interactions.
Synthetically engineered communities are one way of achieving this \citep{momeni:elife:2013}